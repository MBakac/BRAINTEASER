% Paper template for TAR 2022
% (C) 2014 Jan Šnajder, Goran Glavaš, Domagoj Alagić, Mladen Karan
% TakeLab, FER

\documentclass[10pt, a4paper]{article}

\usepackage{tar2023}

\usepackage[utf8]{inputenc}
\usepackage[pdftex]{graphicx}
\usepackage{booktabs}
\usepackage{amsmath}
\usepackage{amssymb}

% Imported by us
\usepackage{xcolor}

% TODO title
\title{Holding BERTs Hands Through Multiple Choices}

% TODO real names
\name{{\color{red}Wumbolo, Tegla}, Nikola Kraljević} 

\address{
University of Zagreb, Faculty of Electrical Engineering and Computing\\
Unska 3, 10000 Zagreb, Croatia\\ 
% TODO real emails
\texttt{\{{\color{red}Wumbolo,Tegla},Nikola.Kraljevic2\}@fer.hr}\\
}
          
         
\abstract{ 
% TODO ovo prepravit, jer nije dovoljno clickbait i to se nece svidit snajderu
Question answering tasks have long been a central challenge within NLP.
This paper presents our approach to tackling the Sentence Puzzle task in the SemEval-2024 
Task 9 competition: "BRAIN-TEASER: A Novel Task Defying Common Sense." Diverging from the 
conventional multi-class classification framework, we introduce a novel pairwise 
comparison methodology.  Instead of forcing the model to choose the correct answer from four 
options, we reformulate the task into multiple decisions, evaluating triplets of 
answers at a time.  This approach, inspired by the efficiency of One-vs-One and One-vs-Rest 
strategies, aims to enhance the model's reasoning capabilities and accuracy
. {\color{red} Our results showcase... ako budu dobri rezultati tu cemo flexat, xD}
}

\begin{document}

\maketitleabstract

\section{Introduction} \label{intro}
Question answering tasks have long been a central challenge within NLP.  The competition, named 
"\textit{SemEval 2024 BRAINTEASER: A Novel Task Defying Common Sense}" described in the paper 
\citep{semeval} is here to push modern models to their limits, putting
them to a quite adversarial setting.  The paper makes a distinction between vertical and lateral
thinking.  Vertical thinking, also known as linear, logical, convergent, is mostly a sequential and 
analytical process, requiring direct memory recall or a few logical steps to come to a sensible 
conclusion. Lateral thinking, also known as "thinking outside the box", is more of a divergent
and creative process, where the question might not make sense when reading it at first and to come
up with an answer you need to explore mutliple angles.  While LLMs show good vertical thinking 
capabiliteis, they are notorious for hallucinating answers. An even more adversarial task for
LLMs is answering questions which require lateral thinking.  State of the art models such as 
ChatGPT show an accuracy of ~60\%, humans show a ~90\% accuracy, while random 
guessing gets close to 25\% accuracy as there are four answer candidates for each question
\footnote[1]{These accuracies are
for one of the subtasks of the competition - "Sentence puzzles", which we will describe in more
detail in the following chapters}.

In this paper, we tackle the second subtask from the competition "Sentence puzzles" with an
alternative approach which we believe will make it easier for the LLM to reason about all the
possible answers. In Section \ref{dataset} we will describe the form of the so called 
"Sentence puzzles".  Our approach and the reasoning behind it will be described in Section 
\ref{approach}  Results of our experiment will be presented in Section \ref{results} and 
Section \ref{conclusion} respectivley.

% TODO spell check, maybe feed to chatgpt to correct it 



\section{Related Work}
Reasoning is a hot topic in NLP, with transformer-based models in the center of attention.
It is no secret that transformer-based models are far from perfect in reasoning \citep{yuan2023}, and that it takes just one session of asking questions about a topic that one is well-versed in to see inconsistencies but also to be amazed at time.
Knowing where a model fails is important and may lead to improvement.  Many benchmarks have been performed with the purpose of challenging models with more intricate reasoning, like "Commonsense QA" \citep{commonsenseQA}, a dataset comprised of questions that are easy for humans and that require no prior knowledge such as a specific document or context, and require just common sense.
Similarly, "BRAINTEASER" \citep{semeval} is a benchmark dataset comprised of questions that are constructed in such a way that it is required to consider multiple approaches when answering them.
As this was a competition dataset, there was previous work that tackled this problem.
\citet{ails-lab} showed promising results, significantly outperforming baselines of 60\% reported in \citep{semeval}.
Their approach consisted of fine-tuning models with a straightforward approach, treating the problem as a multiclass classification task. Additionally they performed a transformation of the problem to a binary classification problem. 
The transformation worked by taking each question with four candidate answers and mapping it to three questions that required a binary label, signalling whether the answer was correct or not \footnote[1]{The fourth candidate answer was always "None of the above" so they discarded it.}.
Their results showed that the transformation was not quite useful, as the same models had somewhat worse accuracies when the transformation was applied.  
This inspired us to explore another transformation of the problem, as we believe that presenting a model with only one candidate answer takes away critical information.
When one is answering a question with candidate answers, it is a good idea to eliminate candidates that are obvious outliers.
That is why we believe that an approach where we force a model to choose between two candidate answers at a time might be helpful, and this is described in detail in Section \ref{approach}. \label{related-work}

\section{Dataset} \label{dataset}
As we've mentioned before the "BRAINTEASER" dataset consists of two types of puzzles, "Sentence puzzles"
and "Word puzzles".  Word puzzles focus more on letter composition, defying the original meaning
of the words at hand, shown in Table \ref{tab:word-puzzle-q}. 
The second type of puzzles we find more interesting, so called "Sentence puzzles", shown in Table
\ref{tab:sentence-puzzle-q}. The three step process of constructing the "BRAINTEASER" dataset
is described in great detail in \citep{semeval}. The process consists of data collection
(web scraping),
distractor sampling and generating reconstruction examples. As some pre-trained models could have
these questions in their training corpus, steps were made to ensure that the questions were novel
for the models.  The data set consists of original data, semantic reconstruction data, context
reconstruction data. The semantic reconstruction used an open-source rephrasing\footnote[2]{https://quillbot.com/} tool and 
human annotators as quality control to rephrase the questions. The context reconstruction process
used GPT-4 to initially shift the context of the question, as well as human annotators. In the 
following sections we will refer to these distinct types of questions as \textbf{Ori.},
\textbf{Sem.} and 
\textbf{Con.} for original questions and semantic/context reconstruction respectivley. Lastly, the
data was split into $507/120/120$ questions for the train/test/validation datasets respectivley.
  

\begin{table}
	\caption{A \textbf{word puzzle} question(sub-task B). The correct answer is in \textbf{bold}.}
	\label{tab:word-puzzle-q}
	\begin{center}
		\begin{tabular}{p{3.5cm}|p{3.5cm}}
			\toprule
			Question                          & Candidates             \\
			\midrule
			What part of London is in France? & \textbf{The letter N.} \\
			                                  & The letter O.          \\
			                                  & The letter L.           \\
			                                  & None of the above.      \\
			\bottomrule
		\end{tabular}
	\end{center}
\end{table}




\begin{table}
	\caption{A \textbf{sentence puzzle} examples(sub-task A). The correct answer is in \textbf{bold}.}
	\label{tab:sentence-puzzle-q}
	\begin{center}
		\begin{tabular}{p{3.5cm}|p{3.5cm}}
			\toprule
			Question                                         & Candidates                                  \\
			\midrule
			A man shaves everyday, yet keeps his beard long. & \textbf{He is a barber.}                    \\
			                                                 & He wants to maintain his appearance.        \\
			                                                 & He wants his girlfriend to buy him a razor. \\
			                                                 & None of the above.                           \\
			\bottomrule
		\end{tabular}
	\end{center}
\end{table}



\section{Pairwise approach} \label{approach}
In this section we will discuss the different approaches to asking multiple-choice questions and describe the training pipeline.
The source code is available on GitHub\footnote[3]{\url{https://github.com/MBakac/BRAINTEASER}}.
\subsection{Reframing multiple choice}
This section will discuss the main point of this paper -- different formulations of a multiple-choice question for BERT.
We will first describe the already explored multiclass and binary classification approaches and introduce the pairwise classification approach.

\subsubsection{Multiclass classification}
For the multiclass classification approach, the question and answers are concatenated and the problem is framed as multiclass classification where the classes are answers $A$, $B$, $C$, and $D$ (None of the above).
The main advantage of this approach is that the model sees all the answers in each pass.

\subsubsection{Binary classification}
In reframing the problem as a binary classification task we take every question and answer pair and label it $1$ if the answer is the correct one, and $0$ otherwise.
When determining the final chosen answer for each question, we group the pairs by question, and, if the model outputs $1$ for only one answer, we chose that answer, otherwise $D$ (None of the above) is chosen.
One advantage of this is that we artificially have more training examples, i.e., question-answer pairs.

\subsubsection{Pairwise approach}
For the new pairwise approach, we transform the questions in the following way.
Each tuple comprised of a question and its candidate answers
$$
    (Q,A,B,C,D)
$$
is transformed into three tuples:
\begin{align*}
       (Q,A,B,D), \\
       (Q,A,C,D), \\
       (Q,B,C,D).
\end{align*}
In case the correct answer was $C$, the ground-truth label in the tuple $(Q,A,B,D)$ is $D$ (None of the above).
For each set of tuples, \emph{votes} for each answer are counted and the answer with the most votes is chosen.
We aim to see if this approach sampling distractors multiple times for each example will yield better results.

\subsection{Training process}
In line with the approach taken by \citep{ails-lab} we load a pre-trained BERT-like model from Hugging Face.
The data evaluation function is modified according to the reformulation and a BERT-like model is fine-tuned to the data.

For this work we decided to use general pre-trained models RoBERTa-large\footnote[4]{\url{https://huggingface.co/FacebookAI/roberta-large}} \citep{roberta} and DeBERTa-v3-base\footnote[5]{\url{https://huggingface.co/microsoft/deberta-v3-base}} \citep{deberta,debertav3} as our goal was to compare different formulations, and was not to achieve the highest score.
For future work, this approach could be taken with less general, fine-tuned BERT-like models.

In our study, we explore binary, pairwise, and multiple classification approaches, utilizing the Hugging Face Transformers library for model fine-tuning. 
For each classification task, we preprocess the data by renaming the column "label" to "labels" and removing irrelevant columns such as "id", "question", "answer", 
and distractors. The models are trained on a dataset tokenized using AutoTokenizer, and set to a format compatible with PyTorch. We employ the
AutoModelForSequenceClassification for binary classification and AutoModelForMultipleChoice for both pairwise and multiple classification tasks. 
The training hyperparameters are set to a batch size of 4, learning rate of 3e-5, and a varying number of epochs: 3 for binary and multiple classification, 
and 1 for pairwise classification. The training process uses the AdamW optimizer and a linear scheduler with no warmup steps. The training arguments include 
evaluating the model every 20 steps, logging every 20 steps, saving checkpoints every 100 steps, and disabling report integrations by setting report\_to=None. 
The models are trained and evaluated on a CUDA-enabled GPU if available. The Trainer class from Hugging Face is employed to manage the training loop, with datasets 
split into training and validation sets, ensuring a structured approach to model training and evaluation. 


\section{Results} \label{results}
\begin{table*}
	\caption{Wide-table caption}
	\label{tab:results-table}
	\begin{center}
		\begin{tabular}{lcccccc}
			\toprule
			\textbf{System}       & \textbf{Original} & \textbf{Semantic} & \textbf{Context} & \textbf{Ori. + Sem.} & \textbf{Ori. + Sem. + Con.} & \textbf{Overall} \\
			\midrule
			\color{gray}Human     & \color{gray}.907  & \color{gray}.907  & \color{gray}.944 & \color{gray}.907     & \color{gray}.889            & \color{gray}.920 \\
			\color{gray}ChatGPT   & \color{gray}.608  & \color{gray}.593  & \color{gray}.679 & \color{gray}.507     & \color{gray}.397            & \color{gray}.627 \\
			\color{gray}RoBERTa-L & \color{gray}.435  & \color{gray}.402  & \color{gray}.464 & \color{gray}.330     & \color{gray}.201            & \color{gray}.434 \\
			\midrule
			\bottomrule
		\end{tabular}
	\end{center}
\end{table*}

\section{Conclusion}
To conclude, we have implemented and tested a new method for reformulating multiple-choice questions.
Our pairwise approach performed better than the binary classification reformulation but worse than the multiclass classification reformulation.
For further research we suggest looking into other possible reformulations, expanding the dataset and using more fine-tuned models. \label{conclusion}

\bibliographystyle{tar2023}
\bibliography{tar2023} 

\end{document}

