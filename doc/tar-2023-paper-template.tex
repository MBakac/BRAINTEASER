% Paper template for TAR 2022
% (C) 2014 Jan Šnajder, Goran Glavaš, Domagoj Alagić, Mladen Karan
% TakeLab, FER

\documentclass[10pt, a4paper]{article}

\usepackage{tar2023}

\usepackage[utf8]{inputenc}
\usepackage[pdftex]{graphicx}
\usepackage{booktabs}
\usepackage{amsmath}
\usepackage{amssymb}

% Imported by us
\usepackage{xcolor}

% TODO title
\title{{\color{red} Our model *shocks* the industry with a novel and innovative approach WOW!}}

% TODO real names
\name{{\color{red}Wumbolo, Tegla}, Nikola Kraljević} 

\address{
University of Zagreb, Faculty of Electrical Engineering and Computing\\
Unska 3, 10000 Zagreb, Croatia\\ 
% TODO real emails
\texttt{\{{\color{red}Wumbolo,Tegla},Nikola.Kraljevic2\}@fer.hr}\\
}
          
         
\abstract{ 
% TODO ovo prepravit, jer nije dovoljno clickbait i to se nece svidit snajderu
This paper presents our approach to tackling the Sentence Puzzle task in the SemEval-2024 
Task 9 competition: "BRAIN-TEASER: A Novel Task Defying Common Sense." Diverging from the 
conventional multi-class classification framework, we introduce a novel pairwise 
comparison methodology.  Instead of forcing the model to choose the correct answer from four 
options, we reformulate the task into multiple binary decisions, evaluating pairs of 
answers at a time.  This approach, inspired by the efficiency of One-vs-One and One-vs-Rest 
strategies, aims to enhance the model's reasoning capabilities and accuracy
. {\color{red} Our results showcase... ako budu dobri rezultati tu cemo flexat, xD}
}

\begin{document}

\maketitleabstract

\section{Introduction}
The competition, named 
"\textit{SemEval 2024 BRAINTEASER: A Novel Task Defying Common Sense}" described in the paper 
\citep{semeval} is here to push modern models to their limits, putting
them to a quite adversarial setting.  The paper makes a distinction between vertical and lateral
thinking.  Vertical thinking, also known as linear, logical, convergent, is mostly a sequential and 
analytical process, requiring direct memory recall or a few logical steps to come to a sensible 
conclusion. Lateral thinking, also known as "thinking outside the box", is more of a divergent
and creative process, where the question might not make sense when reading it at first and to come
up with an answer you need to explore mutliple angles.  While LLMs show good vertical thinking 
capabiliteis, they are notorious for hallucinating answers. An even more adversarial task for
LLMs is answering questions which require lateral thinking.  State of the art models such as 
ChatGPT show an accuracy of ~60\%, humans show a ~90\% accuracy, while random 
guessing gets close to 25\% accuracy as there are four answer candidates for each question
\footnote[1]{These accuracies are
for one of the subtasks of the competition - "Sentence puzzles", which we will describe in more
detail in the following chapters}.

In this paper, we tackle the second subtask from the competition named "Sentence puzzles" with an
alternative approach which we believe will make it easier for the LLM to reason about all the
possible answers. In Section \ref{dataset} we will describe the form of the so called 
"Sentence puzzles".  Our approach and the reasoning behind it will be described in Section 
\ref{approach}  Results of our experiment will be presented in Section \ref{results} and 
Section \ref{conclusion} respectivley.

% TODO spell check, maybe feed to chatgpt to correct it 



\section{Related Work}
Reasoning in NLP is a hot topic, with LLMs in the center of attention.  It is no secret that 
LLMs are far from perfect, and it takes only one session of asking questions about a
topic one is well versed in to see inconsistencies, but also be amazed at times.  Knowing where
a model fails is important and leads to improvement.  Many benchmarks have been made with the
purpose of challenging models with more intricate reasoning, like "Commonsense QA"
\citep{commonsenseQA}, a dataset consisting of questions that are easy for humans and require
no prior knowledge, like a specific document or context, just common sense. Similarly, the "BRAINTEASER"
\citep{semeval} is a benchmark dataset consisting of questions which are constructed
in such a way that it is required to consider multiple approaches when answering them.

As this was a competition dataset, there were some previous works that tackled this problem. 
A team of researchers from the National Technical University of Athens published their submission
\citep{ails-lab} named "\textit{Transformer Models for Lateral Thinking Puzzles}".  They showed promising
results, significantly outperforming baselines of 60\% reported in \citep{semeval}. Their approach consisted
of fine tuning models with a straightforward approach, treating the problem as a
multi-class classification task. Additionally, they performed a
transformation of the problem to a binary classification problem.  The transformation took form of
taking each question with four candidate answers and transforming it to three questions which 
required a binary label, signaling if the answer was correct or not \footnote[1]{The fourth candidate
answer was always "None of the above" so they discarded it.}. Their results showed that the 
transformation was not quite useful, as the same models had somewhat worse accuracies when
the transformation was applied.  

This inspired us to explore another transformation of the problem, as we believe that subjecting
a model to only one candidate answer takes away information.  When one is answering a question
with candidate answers, it is a good idea to eliminate candidates that are somewhat obviously wrong.
That is why we believe an approach where we force a model to choose between two candidate answers
at a time might be helpful, this is described in detail in Section \ref{approach}


\section{Dataset}
As we've mentioned before the "BRAINTEASER" dataset consists of two types of puzzles, "Sentence puzzles"
and "Word puzzles".  Word puzzles focus more on letter composition, defying the original meaning
of the words at hand, shown in Table \ref{tab:word-puzzle-q}. 
The second type of puzzles we find more interesting, so called "Sentence puzzles", shown in Table
\ref{tab:sentence-puzzle-q}. The three step process of constructing the "BRAINTEASER" dataset
is described in great detail in \citep{semeval}. The process consists of data collection
(web scraping),
distractor sampling and generating reconstruction examples. As some pre-trained models could have
these questions in their training corpus, steps were made to ensure that the questions were novel
for the models.  The data set consists of original data, semantic reconstruction data, context
reconstruction data. The semantic reconstruction used an open-source rephrasing\footnote[2]{https://quillbot.com/} tool and 
human annotators as quality control to rephrase the questions. The context reconstruction process
used GPT-4 to initially shift the context of the question, as well as human annotators. In the 
following sections we will refer to these distinct types of questions as \textbf{Ori.},
\textbf{Sem.} and 
\textbf{Con.} for original questions and semantic/context reconstruction respectivley. Lastly, the
data was split into $507/120/120$ questions for the train/test/validation datasets respectivley.
  

\begin{table}
	\caption{A \textbf{word puzzle} question(sub-task B). The correct answer is in \textbf{bold}.}
	\label{tab:word-puzzle-q}
	\begin{center}
		\begin{tabular}{p{3.5cm}|p{3.5cm}}
			\toprule
			Question                          & Candidates             \\
			\midrule
			What part of London is in France? & \textbf{The letter N.} \\
			                                  & The letter O.          \\
			                                  & The letter L.           \\
			                                  & None of the above.      \\
			\bottomrule
		\end{tabular}
	\end{center}
\end{table}




\begin{table}
	\caption{A \textbf{sentence puzzle} examples(sub-task A). The correct answer is in \textbf{bold}.}
	\label{tab:sentence-puzzle-q}
	\begin{center}
		\begin{tabular}{p{3.5cm}|p{3.5cm}}
			\toprule
			Question                                         & Candidates                                  \\
			\midrule
			A man shaves everyday, yet keeps his beard long. & \textbf{He is a barber.}                    \\
			                                                 & He wants to maintain his appearance.        \\
			                                                 & He wants his girlfriend to buy him a razor. \\
			                                                 & None of the above.                           \\
			\bottomrule
		\end{tabular}
	\end{center}
\end{table}



\section{Evaluation Results}
We compared the three described formulations of multiple-choice question answering across two BERT variants and looked at the model accuracy overall as well as specifically for \emph{semantic} and \emph{context} subsets separated for the Sentence puzzle and Word puzzle subtasks (Table \ref{tab:sentence-results-table} and Table \ref{tab:word-results-table}). 
Instance- and group-based performance metrics for our models as well as the human, ChatGPT and RoBERTa-L baselines are presented.
The results are presented in terms of accuracy.
The individual metric columns \emph{Original}, \emph{Semantic} and \emph{Context} in the tables represent the results of respective reconstructions in the dataset, while the group metric columns \emph{Ori. + Sem} and \emph{Ori. + Sem. + Con.} represent accuracy scores across multiple reconstructions of the same question.

\subsection{Zero accuracy in the binary classification formulation} \ % malo lijepše sve to 
A clear outlier in Table \ref{tab:word-results-table} are the accuracy results in the binary classification formulation of multiple choice.
This is due to the \emph{harsh} criteria for deciding on an answer with this approach where only one out of three binary questions has to be an \emph{A}, \emph{B} or \emph{C} and the other two $D$ \emph{(None of the above)}.
This, combined with the rather small dataset, and also the word puzzle task being harder overall (our models yielded worse overall accuracy scores for this task), meant that there were no cases where this approach worked, hence the accuracy is zero.
Furthermore, we note that the models have been trained to have a strong bias toward answering negatively to all instances.
This is partially caused by the fact that the dataset used to train the models for binary classification is imbalanced, with a ratio of 1:3 between positive and negative instances.
This is a significant issue as the models are not able to learn the patterns of the positive instances, which are the ones that we are interested in finding out.
The same poor results for this formulation were obtained by Panagiotopoulos et. al. \citep{ails-lab}.

\subsection{Pairwise approach evaluation}
Our pairwise approach showed performance on par with the baseline ChatGPT and RoBERTa-L models\citep{semeval} for individual metrics, and better performance for group metrics for both subtasks.
Comparing our approach with the multiclass classification approach, our approach performed worse across both subtasks.
Our approach showed better performance than the binary classification approach that showed miserable results across the board.
Evidently, training the model on questions and all answers (i.e., using the multiclass approach) yields the best results.

\section{Conclusion}
To conclude, we have implemented and tested a new method for reformulating multiple-choice questions.
Our pairwise approach performed better than the binary classification reformulation but worse than the multiclass classification reformulation.
For further research we suggest looking into other possible reformulations, expanding the dataset and using more fine-tuned models.

\section{Acknowledgements}
Acknowledgements: \citep{semeval} semeval, \citep{ails-lab} ails-lab




\bibliographystyle{tar2023}
\bibliography{tar2023} 

\end{document}

