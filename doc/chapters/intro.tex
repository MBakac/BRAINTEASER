Question answering tasks have long been a central challenge within NLP.  The competition, named 
"\textit{SemEval 2024 BRAINTEASER: A Novel Task Defying Common Sense}" described in the paper 
\citep{semeval} is here to push modern models to their limits, putting
them to a quite adversarial setting.  The paper makes a distinction between vertical and lateral
thinking.  Vertical thinking, also known as linear, logical, convergent, is mostly a sequential and 
analytical process, requiring direct memory recall or a few logical steps to come to a sensible 
conclusion. Lateral thinking, also known as "thinking outside the box", is more of a divergent
and creative process, where the question might not make sense when reading it at first and to come
up with an answer you need to explore mutliple angles.  While LLMs show good vertical thinking 
capabiliteis, they are notorious for hallucinating answers. An even more adversarial task for
LLMs is answering questions which require lateral thinking.  State of the art models such as 
ChatGPT show an accuracy of ~60\%, humans show a ~90\% accuracy, while random 
guessing gets close to 25\% accuracy as there are four answer candidates for each question
\footnote[1]{These accuracies are
for one of the subtasks of the competition - "Sentence puzzles", which we will describe in more
detail in the following chapters}.

In this paper, we tackle the second subtask from the competition named "Sentence puzzles" with an
alternative approach which we believe will make it easier for the LLM to reason about all the
possible answers. In Section \ref{dataset} we will describe the form of the so called 
"Sentence puzzles".  Our approach and the reasoning behind it will be described in Section 
\ref{approach}  Results of our experiment will be presented in Section \ref{results} and 
Section \ref{conclusion} respectivley.

% TODO spell check, maybe feed to chatgpt to correct it 

