As we have mentioned before, the "BRAINTEASER" dataset consists of two types of puzzles, \textit{Sentence puzzles} and \textit{Word puzzles}.
Word puzzles focus more on letter composition, defying the original meaning of the words at hand, as shown in Table \ref{tab:word-puzzle-q}.
The second type of puzzles, so-called \textit{Sentence puzzles}, are shown in Table \ref{tab:sentence-puzzle-q}.
The three-step process of constructing the "BRAINTEASER" dataset is described in great detail in \citep{semeval}.
The process consists of collecting data (Internet scraping), sampling distractors and generating reconstruction examples.
As some pre-trained models could have included these questions in their training corpus, steps were taken to ensure that the questions were novel to the models \cite{fine-tune}.
The dataset consists of the original data, the semantic reconstruction data, and the context reconstruction data.
The semantic reconstruction used an open-source rephrasing tool\footnote[2]{\url{https://quillbot.com/}} and human annotators as quality control to rephrase the questions. The context reconstruction process used GPT-4, as well as human annotators, to initially shift the context of the question.
In the following sections we will refer to these distinct types of questions as \textbf{Ori.}, \textbf{Sem.} and \textbf{Con.} for original questions and semantic and context reconstructions.
Finally, the data was split into $507/120/120$ questions for the \textit{train}/\textit{test}/\textit{validation} datasets respectively.
\begin{table}
	\caption{A \textbf{word puzzle} question (sub-task B). The correct answer is in bold.}
	\label{tab:word-puzzle-q}
	\begin{center}
		\begin{tabular}{p{3.5cm}|p{3.5cm}}
			\toprule
			Question               & Candidates             \\
			\midrule
			What part of London is & \textbf{The letter N.} \\
			in France?             & The letter O.          \\
			                       & The letter L.          \\
			                       & None of the above.     \\
			\bottomrule
		\end{tabular}
	\end{center}
\end{table}

\begin{table}
	\caption{A \textbf{sentence puzzle} examples (sub-task A). The correct answer is in bold.}
	\label{tab:sentence-puzzle-q}
	\begin{center}
		\begin{tabular}{p{3.5cm}|p{3.5cm}}
			\toprule
			Question                  & Candidates                                  \\
			\midrule
			A man shaves everyday,    & \textbf{He is a barber.}                    \\
			yet keeps his beard long. & He wants to maintain his appearance.        \\
			                          & He wants his girlfriend to buy him a razor. \\
			                          & None of the above.                          \\
			\bottomrule
		\end{tabular}
	\end{center}
\end{table}

