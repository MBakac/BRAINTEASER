As we've mentioned before the "BRAINTEASER" dataset consists of two types of puzzles, "Sentence puzzles"
and "Word puzzles".  Word puzzles focus more on letter composition, defying the original meaning
of the words at hand, shown in Table \ref{tab:word-puzzle-q}. 
The second type of puzzles we find more interesting, so called "Sentence puzzles", shown in Table
\ref{tab:sentence-puzzle-q}. The three step process of constructing the "BRAINTEASER" dataset
is described in great detail in \citep{semeval}. The process consists of data collection
(web scraping),
distractor sampling and generating reconstruction examples. As some pre-trained models could have
these questions in their training corpus, steps were made to ensure that the questions were novel
for the models.  The data set consists of original data, semantic reconstruction data, context
reconstruction data. The semantic reconstruction used an open-source rephrasing\footnote[2]{https://quillbot.com/} tool and 
human annotators as quality control to rephrase the questions. The context reconstruction process
used GPT-4 to initially shift the context of the question, as well as human annotators. In the 
following sections we will refer to these distinct types of questions as \textbf{Ori.},
\textbf{Sem.} and 
\textbf{Con.} for original questions and semantic/context reconstruction respectivley. Lastly, the
data was split into $507/120/120$ questions for the train/test/validation datasets respectivley.
  

\begin{table}
	\caption{A \textbf{word puzzle} question(sub-task B). The correct answer is in \textbf{bold}.}
	\label{tab:word-puzzle-q}
	\begin{center}
		\begin{tabular}{|p{3.5cm}|p{3.5cm}|}
			\toprule
			Question                          & Candidates             \\
			\midrule
			What part of London is in France? & \textbf{The letter N.} \\
			                                  & The letter O.          \\
			                                  & The letter L.           \\
			                                  & None of the above.      \\
			\bottomrule
		\end{tabular}
	\end{center}
\end{table}




\begin{table}
	\caption{A \textbf{sentence puzzle} examples(sub-task A). The correct answer is in \textbf{bold}.}
	\label{tab:sentence-puzzle-q}
	\begin{center}
		\begin{tabular}{|p{3.5cm}|p{3.5cm}|}
			\toprule
			Question                                         & Candidates                                  \\
			\midrule
			A man shaves everyday, yet keeps his beard long. & \textbf{He is a barber.}                    \\
			                                                 & He wants to maintain his appearance.        \\
			                                                 & He wants his girlfriend to buy him a razor. \\
			                                                 & None of the above.                           \\
			\bottomrule
		\end{tabular}
	\end{center}
\end{table}

